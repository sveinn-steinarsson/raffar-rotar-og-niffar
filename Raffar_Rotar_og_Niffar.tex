\documentclass[ebook,11pt,oneside,openany]{memoir}

\setlrmarginsandblock{2.5cm}{2.5cm}{*}
\setulmarginsandblock{2.5cm}{2.5cm}{*}
\checkandfixthelayout

\pagestyle{plain}

\usepackage[utf8]{inputenc}
\usepackage[T1]{fontenc}

\usepackage[icelandic]{babel}
\selectlanguage{icelandic}

\usepackage{fouriernc}
\usepackage[pdfusetitle]{hyperref}

\usepackage{endnotes}
\renewcommand{\notesname}{Útskýringar}

\title{Raffar, Rótar og Niffar}
\date{22. mars 2015}
\author{Sveinn Steinarsson}

\begin{document}

\maketitle
\thispagestyle{empty}

\vfill
\begin{center}
{\footnotesize
Gefið út undir leyfi
\\\href{http://creativecommons.org/licenses/by/4.0/}{Creative Commons Vísun til höfundar 4.0 Alþjóðlegt}.
}
\end{center}

\newpage

\setcounter{page}{1}

Fyrir langa löngu síðan var ekkert til. Engin Jörð, engin Sól, engar stjörnur, ekki neitt! En allt í einu varð risastór sprenging og í henni varð alheimurinn til.\endnote{Upphaf alheimsins, nefndur Miklihvellur, er talinn hafa átt sér stað fyrir 13,8 milljörðum ára. Vísindamönnum ber ekki alveg saman með hvaða hætti Miklihvellur kom til þótt sumar kenningar hafi fengið meiri meðbyr en aðrar.} Hann var samt mjög skrítinn strax eftir sprenginguna og ólíkur því sem hann er í dag. Í upphafi var alheimurinn nefnilega alveg rosalega heitur og algjört myrkur ríkti í honum. Svo voru bara til þrjár tegundir af pínulitlum eindum, svo litlar að þær voru ósýnilegar. Eindir eru eins og pínulitlir kubbar, af mörgum stærðum og gerðum, sem er hægt að raða saman á margan hátt og búa til margt úr þeim. Í byrjun voru samt bara til þrjár tegundir einda sem eru kallaðar Raffar, Rótar og Niffar. Raffarnir voru alveg pínulitlir, oftast í vondu skapi og afar neikvæðir. Svo voru það Rótarnir sem voru miklu stærri, mjög jákvæðir og alltaf brosandi. Loks voru það Niffarnir; þeir voru svipað stórir og Rótarnir, en voru í hvorki góðu né vondu skapi og var alveg sama um flest allt.\endnote{Verið er að tala um neikvætt hlaðnar rafeindir, jákvætt hlaðnar róteindir og óhlaðnar nifteindir.}

Það var ekki mikið fyrir þessar nýju eindir að gera því það var ennþá alltof heitt til að leika sér eitthvað. Eftir smá stund fór alheimurinn aðeins að kólna og þá tók einn Raffinn upp á því að fara leika við einn Rótann. Hann byrjaði að fara í kringum Rótann alveg rosalega hratt, svo hratt að Raffinn sást varla lengur. Hinir Raffarnir horfðu á og fannst þetta voðalega sniðugt, þannig að fleiri og fleiri fóru að leika þetta eftir. Brátt voru eiginlega allir Raffarnir farnir að þjóta í kringum næstum alla Rótana og stundum í kringum bæði Róta og Niffa í einu.\endnote{Rafeindir fara á braut um róteindir og nifteindir og mynda fyrstu frumeindirnar, mest vetni og helín. Þetta gerðist 300 þúsund árum eftir Miklahvell.}

\bigskip

Á þessum tíma frá því að alheimurinn varð til hafði hann aldeilis breyst mikið. Hann stækkaði hratt og var ekki nærri því eins heitur og í upphafi. Raffarnir héldu samt áfram uppteknum hætti að fara í kringum Rótana. Þeir urðu svo óaðskiljanlegir að farið var að kalla þá einu nafni. Þegar einn Raffi var að fara í kringum einn Rótann voru þeir saman kallaðir Vetti. Svo þegar tveir Raffar fóru í kringum tvo Róta voru þeir allir kallaðir saman Helli. Það má í raun segja að það hefði verið allt fullt af Vettum og einhverjum Hellum líka, en þó mun færri. Bæði Vettarnir og Hellarnir vildu helst líka vera alltaf saman í hópum og leiddust að vera einir. Því stærri sem hópurinn var, því betra. Í fyrstu voru hóparnir litlir en síðan fóru þeir að renna saman og stækka. Sumir hóparnir urðu svo stórir að þótt hver og einn Vetti og Helli væri svo lítill að ekki var hægt að sjá hann, þá voru hóparnir sem þeir tilheyrðu alveg ofboðslega stórir. Þessir risastóru hópar líktust helst risastórum skýjum, nema þetta voru engin venjuleg ský á himni, heldur afar undarleg ský sem voru svo stór að það sást varla hvar þau byrjuðu og hvar þau enduðu.\endnote{Með tilkomu vetnis og helíns fer þyngdaraflið að segja til sín og risastór gasský myndast úr þessum efnum sem síðan mynda fyrstu stjörnurnar og stjörnuþokurnar. Þetta er talið hafa gerst einum milljarði ára eftir Miklahvell.}

Með tímanum fóru þessi risastóru ský að dragast saman. Ekki vegna þess að Vettarnir og Hellarnir fóru að tínast úr þeim, heldur af því þá langaði að vera enn nær hvor öðrum og því nær sem þeir voru, því þrengra var fyrir alla. Smátt og smátt urðu þessi risastóru ský minni og þéttari og eftir því sem þau minnkuðu meira urðu þau meira í laginu eins og stórar kúlur.

Djúpt inni í þessum risakúlum var mjög lítið pláss fyrir Vettana og Hellana. Engu að síður reyndu allir að fara rosalega hratt, en það var svo erfitt því allir voru alltaf að klessa á hvern annan. Vettarnir klesstu oft svo fast á hvern annan að þeir hreinlega festust saman. Við það hitnaði þeim alveg svakalega. Það varð svo heitt að þessar stóru kúlur urðu alveg glóandi og hituðu og lýstu upp allt í kringum sig. Í dag er oft hægt að sjá svona kúlur upp á himninum þegar dimmt er úti. Nema þær eru ekki kallaðar kúlur, heldur stjörnur og virðast bara vera litlir punktar því þær eru svo rosalega langt í burtu.\endnote{Stjörnur halda áfram að þéttast vegna þyngdarkraftsins, þar til hitinn inni í þeim er orðinn nokkrar milljónir gráða. Þá byrjar kjarnasamruni vetnis sem myndar helín og losar gífurlega orku.}

\bigskip

Stjörnurnar endast samt ekki að eilífu. Þegar tveir Vettar klessa fast saman verður úr þeim einn Helli. Eftir nokkuð langan tíma innan stjarnanna verða allir Vettarnir búnir að klessast saman og bara Hellar eftir. Þegar það eru engir Vettar eftir fara Hellarnir að klessa hver á annan og verða þá til nýjar og stærri eindir sem fara jafnvel líka að klessa hver á aðra. Við þetta verður stjarnan stærri og stærri og stundum kemur það fyrir að hún hreinlega springur. Þá hendast flestar eindirnar sem voru í stjörnunni út í geim og koma jafnvel saman í annað risastórt ský sem verður stundum eftir langan tíma að annarri stjörnu.\endnote{Þegar stjörnur hafa eytt öllu nothæfu vetni sem er inn í þeim fer í gang kjarnasamruni með helíni og þyngri frumefnum. Þannig myndast enn þyngri frumefni. Ef stjarnan er nógu stór springur hún í sumum tilfellum og skilar þessum frumefnum aftur út í umhverfið. Þær stjörnur kallast sprengistjörnur. Inni í stjörnum og sprengistjörnum myndast öll þau frumefni sem ekki urðu til í Miklahvelli. Stjörnur eru því oft kallaðar efnaverksmiðjur alheimsins og við erum í raun búin til úr stjörnuefni.}

Sólin okkar er líka stjarna, nema hún er svo nálægt að okkur finnst hún vera öðruvísi og stærri en aðrar stjörnur. Hún varð líka til úr risastóru skýi af ósýnilegum eindum eins og Vettum og Hellum, alveg eins og allar hinar stjörnurnar. Hún er líka rosalega heit og björt því Vettarnir, sem eru inn í henni, eru sífellt að klessa hver á annan og við það hitnar Sólin.\endnote{Fyrir um það bil 5 milljörðum ára byrjaði gríðarlega stórt ský úr gasi og ryki að koma saman og mynda sólina.}

Ekki gátu þó allar eindirnar í risastóra skýinu sem Sólin okkar var gerð úr, orðið hluti af henni. Skýið var svo stórt að sumar eindirnar í því voru bara of langt í burtu til að geta verið með í að búa til Sólina. Eindirnar sem voru eftir vildu samt vera nálægt hvor annarri, svo þær fóru að hópa sig saman í marga litla hópa. Litlu hóparnir urðu þéttari og þéttari svo úr varð fullt af pínulitlum kornum. Svo fóru þessi litlu korn að rekast á og festast saman. Sumir hóparnir innihéldu svo mörg lítil korn að þeir voru orðnir að stærðarinnar hnullungum, jafnvel stærri en hús. Það var ekki nóg með það, heldur klesstu þessir hnullungur líka hvern á annan og festust stundum saman. Þeir minntu helst á risavaxna kringlótta steina og fóru hringi í kringum Sólina, sem var í miðjunni. Þessir steinar urðu þó aldrei nærri því jafn stórir og Sólin og útaf því þeir urðu ekki jafn stórir var ekki eins þröngt inn í þeim. Eindirnar sem þessir steinar voru úr þurftu því aldrei að klessa hver á aðra eins og inni í Sólinni og öllum hinum stjörnunum.

Það má segja að Jörðin sem við búum á sé í raun risastór hnöttóttur steinn og fer hring eftir hring í kringum Sólina. Svipað má segja um hinar reikistjörnurnar eins og Venus og Mars. Tunglið okkar er líka kringlóttur steinn, bara ekki eins stór og Jörðin. Tunglið fer samt ekki í kringum Sólina heldur fer Tunglið í kringum Jörðina, hring eftir hring.

\bigskip

Fyrst eftir að Jörðin varð til var ekkert fólk á henni. Ekki nóg með það, heldur voru engin dýr, engin tré, ekki einu sinni vatn. Jörðin var bara brennandi heitur, risastór hnöttur sem fór hringi í kringum Sólina.\endnote{Jörðin varð til fyrir u.þ.b. 4,5 milljörðum ára. Efnisagnir á sporbaug um Sólina sem höfðu orðið afgangs eftir myndun hennar komu saman vegna þyngdaraflsins og mynduðu stærri hluti. Varðandi uppruna Tunglsins er vinsælasta tilgátan sú að önnur frumreikistjarna á stærð við Mars lenti í árekstri við frumjörðina og við það þeyttist efni á sporbaug um Jörðina sem síðan hafi komið saman og myndað Tunglið.}

Í langan tíma gerðist ósköp lítið þar til Jörðin kólnaði lítillega. Þá fór vatn að renna og höf mynduðust. Í þessu vatni var auðvitað fullt af allskonar eindum. Nokkrar eindir í vatninu reyndu að finna upp einhverja skemmtilega leiki. Þær prófuðu að halda sig þétt saman og mynda stóra hópa af mörgum mismunandi tegundum einda. Oftast gáfust þær upp en í eitt skipti datt þeim rosalega sniðugur leikur í hug. Þegar vissar tegundir af eindum röðuðu sér saman á ákveðinn hátt, tókst þeim að búa til fyrstu lifandi veru Jarðar. Lifandi verur eru öðruvísi en eindirnar því verurnar þurfa að borða; Þær verða til og þær deyja, ólíkt eindunum sem þurfa aldrei að borða og deyja aldrei. Samt eru allar lifandi verur, eins og allt í alheiminum, úr þessum grunneindum: Röffum, Rótum og Niffum.\endnote{Stór hluti þess vatns sem er á Jörðinni hefur líklega komið úr halastjörnum og vatnsríkum loftsteinum sem skullu á Jörðina í barnæsku sólkerfisins. Talið er að fyrstu einföldu örverurnar hafi komið til sögunnar á Jörðinni fyrir meira en 3,5 milljörðum árum síðan. Uppruni þeirra er enn ekki fullkomlega ljós og tilgátur eru uppi um að líf á Jörðinni hafi jafnvel kviknað annars staðar og borist hingað með loftsteinum.}

Til þess að gleyma ekki hvernig átti að búa til lifandi verur, komu eindirnar sér saman um að alltaf yrðu einhverjar eindir inni í sérhverri lifandi veru, sem hefðu það hlutverk að muna hvernig ætti að búa til veruna. Þær þurftu að muna hvaða eindir þurfi til og hvernig þeim er raðað saman. Þessar eindir, sem fengu það hlutverk að muna, eru kallaðar Dinnar. Þær eru afskaplega mikilvægar, því án þeirra myndi gleymast hvernig ætti að búa til lifandi verur.\endnote{Talið er að fyrsta erfðaefnið hafi verið kjarnasýran RNA sem hafi seinna þróast í DNA. Erfðaefni ákvarða gerðir prótína sem stjórna eiginleikum fruma.}

\bigskip

Fyrstu lifandi verurnar voru svo litlar að það var ekki hægt að sjá þær með berum augum. Þær bárust um í vatninu í von um að finna eitthvað til að borða því allar lifandi verur þurfa næringu. Stundum, þegar nóg var til að borða, bjuggu verurnar til aðrar verur með því að skipta sér í tvennt. Dinnarnir vissu nefnilega hvernig átti að búa til nýjar verur og áttu að passa að þær væru alltaf nákvæmlega eins.

Það kom samt stundum fyrir að Dinnarnir gerðu smá mistök þegar veran sem þeir voru í var að skipta sér í tvennt. Þá urðu þessar tvær verur ekki alveg nákvæmlega eins. Önnur varð kannski aðeins stærri en hin eða kannski gat önnur hreyft sig hraðar. Það var ekki alltaf slæmt, því stundum var betra að verur gætu hreyft sig hraðar, þá voru þær nefnilega fljótari en hægari verurnar að borða matinn. Verurnar sem voru hægari urðu að hætta að skipta sér því þær fengu aldrei nóg að borða. Fljótari verurnar voru alltaf á undan að borða allan matinn. Eftir smá tíma voru hægu verurnar alveg horfnar og bara fljótu verurnar eftir. Þó var ekki alltaf svo að hægari verurnar hættu að vera til. Þær fóru stundum bara annað, þangað sem þær gátu haft matinn sinn í friði fyrir fljótu verunum. Jörðin er nefnilega mjög stór og alveg nóg pláss í sjónum og vötnunum svo verurnar þurftu ekki allar að vera á sama staðnum eða borða það sama.\endnote{Stundum þegar frumur skipta sér, verða stökkbreytingar sem geta haft í för með sér nýja eiginleika. Þessir eiginleikar geta veitt nýrri frumu forskot í því umhverfi sem hún lifir í. Þetta er stundum nefnt náttúruval, sem er undirstaðan í þróunarkenningu Darwins. Þeir hæfustu lifa af og hinir sem eru minna hæfir verða undir í lífsbaráttunni.}

Eftir nokkurn tíma urðu sumar verur samt leiðar á því að vera alltaf einar. Því ákváðu einhverjar verur að gerast perluvinir, standa saman og hjálpast að. Við það varð til stærri vera sem var úr fjölmörgum minni verum. Þessar minni verur höfðu síðan ákveðnu hlutverki að gegna inni í stóru verunni. Sumar sáu til dæmis um hreyfingar og aðrar sáu um að melta matinn.\endnote{Verið að lýsa því þegar fyrstu fjölfrumungarnir urðu til úr einfrumungum.}

Þessar stóru verur skiptu sér ekki í tvennt eins og litlu verurnar. Heldur hittust þessar stóru verur og blönduðu Dinnunum sínum saman. Dinnarnir voru litlu eindirnar inni í öllum lifandi verum sem höfðu það hlutverk að muna hvernig ætti að búa til nýjar lifandi verur. Með því að blanda saman Dinnum urðu til nýjar stórar verur sem voru blanda af tveimur stórum verum. Ef til dæmis önnur stóra veran var rosalega fljót og hin stóra veran gríðarlega sterk, þá gátu þær blandað saman Dinnum og búið til nýja stóra veru sem var bæði fljót og sterk.\endnote{Fyrir rúmlega einum milljarði ára er talið að kynæxlun hafi byrjað og það hafi hraðað mjög allri þróun. Erfðaefninu er samt ekki beinlínis blandað saman. Flóknar lífverur hafa litningana í pörum og við kynæxlun fær afkvæmið litninga í hverju pari frá sitthvoru foreldrinu.}

Með tímanum urðu þessar stóru verur, sem voru úr mörgum litlum verum, stærri og stærri. Þær urðu svo stórar að loks hefði verið hægt að auðveldlega sjá þær með berum augum. Svo stækkuðu þær enn meira og fóru að líkjast meira þeim dýrum sem við þekkjum í dag. Öll dýrin voru samt ennþá ofan í vatni þannig að þau sem urðu stærst fóru að líkjast meira fiskum og öðrum sjávardýrum. Þegar tíminn leið fengu nokkrir fiskar pínulitla fætur og byrjuðu að skríða upp úr sjónum. Þegar þeir voru komnir upp á land vildu einhverjir ekki fara aftur í sjóinn og héldu sig bara á landinu. Dýrin sem fóru ekki aftur í sjóinn hættu þá að líkjast fiskum og fóru að líkjast meira eðlum og öðrum dýrum sem eru betur til þess fallin að búa á þurru landi.\endnote{Fiskar eru taldir hafa komið til sögunnar fyrir um 500 milljón ára. Seinna, eða fyrir um 363 milljón árum, er talið að fyrstu ferfætlingarnir hafi þróast og það hafi hjálpað þeim að lifa á þurru landi. Skordýr voru líklega þegar komin upp á land og plöntur komið til sögunnar sennilega um 100 milljón árum fyrr.}

\bigskip

Lífið var ekki alltaf auðvelt fyrir dýrin. Ekki var alltaf til nægur matur fyrir alla og svo komu líka upp mörg önnur vandamál. Til dæmis gat veðrið breyst skyndilega og það raskaði ýmsu. Einu sinni skall risastór loftsteinn á Jörðina og olli geysilega mikilli sprengingu. Þá breyttist veðrið mikið og varð miklu kaldara en áður. Um það leyti voru það risaeðlurnar sem réðu ríkjum og flest litlu dýrin voru mjög hrædd við þær. En risaeðlurnar höfðu engan feld og dóu því allar úr kulda.\endnote{Um 70\% lífvera, þar á meðal risaeðlurnar, dóu út fyrir um 65 milljónum ára. Orsökin er talin liggja í miklum náttúruhamförum af völdum loftsteins eða jafnvel voldugs eldgoss. Það hafði trúlega í för með sér mikla kólnun vegna rykagna í lofthjúpnum. Í sögu Jarðarinnar hafa svona skyndilegar loftslagsbreytingar oft valdið útdauða afar margra dýrategunda.}

Þegar risaeðlurnar voru útdauðar gátu litlu dýrin stækkað og fjölgað sér hratt,  án þess að eiga á hættu að vera étin af risaeðlum. Þá urðu til margar nýjar tegundir af dýrum. Ein dýrategundin líktist öpum sem tóku að ganga á tveimur fótum. Kannski til að komast í annað tré lengra í burtu með meiri mat. Með tímanum þurftu þeir að ganga lengra og sumir hættu alveg að búa í trjám. Þá fóru þeir að líkjast mannfólkinu eins og við lítum út í dag. Við erum í raun fjarskyld öpum og líka öllum öðrum lifandi verum á Jörðinni.

\bigskip

Svona varð alheimurinn til og við sömuleiðis, því við erum hluti af honum. Þetta er samt bara smá brot af allri sögunni því öll sagan um hvernig alheimurinn varð til er rosalega löng og flókin. Svo vitum við alls ekki allt um alheiminn en erum alltaf að læra meira og meira.

Í stuttu máli má samt segja að alheimurinn hafi orðið til í risastórri sprengingu fyrir langa löngu. Svo varð Sólin okkar til og þar næst Jörðin, en allt gerðist þetta á geysilega löngum tíma. Þegar Jörðin var glæný var ekkert líf á henni, engin dýr og engar plöntur, en þetta koma allt smátt og smátt með tímanum. Stuttu eftir upphaf alheimsins voru aðeins til þrjár tegundir einda: Raffar, Rótar og Niffar. Í dag eru ennþá aðeins til þessar sömu þrjár tegundir einda því allt í alheiminum er gert úr þeim. Við erum gerð úr þeim og líka allt í kringum okkur eins og dýrin, fjöllin og stjörnurnar. Þessar eindir eru til í alvöru nema þær heita ekki Raffar, Rótar og Niffar eins og í þessari sögu, heldur heita þær rafeindir, róteindir og nifteindir.\endnote{Róteindir og nifteindir eru tæknilega ekki grunneindir heldur gerðar úr jafnvel enn minni eindum sem nefnast kvarkar. Til einföldunar er ekkert minnst á kvarka í þessari sögu né aðrar öreindir sem vitað er um í hinu viðtekna líkani í öreindafræði nútímans.}


\newpage 
\pagestyle{empty}

\theendnotes

\end{document}